\documentclass[11pt, english]{article}
\usepackage[utf8]{inputenc} %packages
\usepackage[T1]{fontenc}
\usepackage{babel}
%opening
\title{CSE3211: Operating System Assignment 0} %title of the report
% author name: you and your partner
\author{Al Md. Aladin\\
	student ID: 2015-516-820 \& roll number: FH-59
	\and
Sharear Bin Amin\\
student ID: 2015-016-816 \& roll number: FH-55
}
\date{August 13, 2018} %change date as requires
\begin{document}
\maketitle
\section{Questions \& Answers}
\textit{Q1. What is the vm system called that is configured for assignment 0?}\newline % single new line
\textbf{Answer: dumbvm from kern/arch/mips/conf/conf.arch
	}\\ \\ % \\ \\ two new line
\textit{Q2. Which register number is used for the stack pointer (sp) in OS/161?}\newline
\textbf{Answer: \#define sp \$29 /* stack pointer */  from kern/arch/mips/include/kern/regdefs.h
	}\\ \\
\textit{Q3. What bus/busses does OS/161 support?}\newline
\textbf{Answer: LAMEbus      from kern/arch/sys161/include/bus.h
	}\\ \\
\textit{Q4. What is the difference between splhigh and spl0?}\newline
\textbf{Answer: splhigh()    sets IPL to the highest value, disabling all interrupts.
	spl0()       sets IPL to 0, enabling all interrupts.
	from os161-ASST0/kern/include/spl.h
	}\\ \\
\textit{Q5. Why do we use typedefs like u\_int32\_t instead of simply saying "int"?}\newline
\textbf{Answer: To make sure that we really get a 32-bit unsigned integer
	(unsigned int depends on the platform)
	from kern/arch/mips/include/types.h
	}\\ \\
\textit{Q6. What must be the first thing in the process control block?}\newline
\textbf{Answer: pcb\_switchstack must be the first thing in the process control block.
	}\\ \\
\textit{Q7. What does splx return?}\newline
\textbf{Answer: The old interrupt state (an integer)
	from os161-ASST0/kern/include/spl.h
}\\ \\
\textit{Q8. What is the highest interrupt level?}\newline
\textbf{Answer: \#define IPL\_HIGH   1
	from  os161-ASST0/kern/include/spl.h
}\\ \\
\textit{Q9. What function is called when user-level code generates a fatal fault?}\newline % single new line
\textbf{Answer: void kill\_curthread(vaddr\_t epc, unsigned code, vaddr\_t vaddr)
	from os161-ASST0/kern/arch/mips/mips/trap.c
}\\ \\
\textit{Q10. How frequently are hardclock interrupts generated?}\newline
\textbf{Answer: 100 times a second   \#define HZ  100
	from kern/include/clock.h
}\\ \\ % \\ \\ two new line
\textit{Q11. What functions comprise the standard interface to a VFS device?}\newline
\textbf{Answer: vfs\_setcurdir, vfs\_clearcurdir, vfs\_getcurdir, vfs\_sync, vfs\_getroot,
	vfs\_getdevname, vfs\_lookup, vfs\_lookparent, vfs\_open, vfs\_close,
	vfs\_readlink, vfs\_symlink, vfs\_mkdir, vfs\_link, vfs\_remove, vfs\_rmdir,
	vfs\_rename, vfs\_chdir, vfs\_getcwd, vfs\_bootstrap, vfs\_initbootfs,
	vfs\_setbootfs, vfs\_clearbootfs, vfs\_adddev, vfs\_addfs, vfs\_mount,
	vfs\_unmount, and vfs\_unmountall
	from  os161-ASST0/kern/include/vfs.h
}\\ \\
\textit{Q12. How many characters are allowed in a volume name?}\newline
\textbf{Answer: \#define SFS\_VOLNAME\_SIZE  32  /* max length of volume name */
	from kern/include/kern/sfs.h
}\\ \\
\textit{Q13. How many direct blocks does an SFS file have?}\newline
\textbf{Answer: \#define SFS\_NDIRECT       15            /* \# of direct blocks in inode */
	from kern/include/kern/sfs.h
}\\ \\
\textit{Q14. What is the standard interface to a file system (i.e., what functions must you
	implement to implement a new file system)?}\newline
\textbf{Answer: fsop\_sync, fsop\_getvolname, fsop\_getroot, fsop\_umount
	from kern/include/fs.h
}\\ \\
\textit{Q15. What function puts a thread to sleep?}\newline
\textbf{Answer: Void wchan\_sleep(struct wchan *wc, struct spinlock *lk)
	from kern/thread/thread.c
	}\\ \\
\textit{Q16. How large are OS/161 pids?}\newline
\textbf{Answer: typedef int32\_t pid\_t; /* Process ID */
	32 bits / 4 bytes from kern/include/kern/types.h
	}\\ \\
\textit{Q17. What operations can you do on a vnode?}\newline % single new line
\textbf{Answer: vop\_eachopen,vop\_ reclaim,vop\_ read, vop\_readlink, vop\_getdirentry, vop\_write, vop\_ioctl, vop\_stat, vop\_gettype, vop\_tryseek, vop\_fsync, vop\_mmap, vop\_truncate, vop\_namefile, vop\_creat, vop\_symlink, vop\_mkdir, vop\_link, vop\_remove, vop\_rmdir, vop\_rename, vop\_lookup, vop\_lookparent
	from kern/include/vnode.h
	}\\ \\ % \\ \\ two new line
\textit{Q18. What is the maximum path length in OS/161?}\newline
\textbf{Answer: * Longest full path name */ \#define PATH\_MAX 1024 
	from kern/include/kern/limits.h
	}\\ \\
\textit{Q19. What is the system call number for a reboot?}\newline
\textbf{Answer: The system call number for a reboot is 119.
	}\\ \\
\textit{Q20. Where is STDIN\_FILENO defined?}\newline
\textbf{Answer: \#define STDIN\_FILENO 0 /* Standard input */ 
	from kern/include/kern/unistd.h
	}\\ \\
\textit{Q21. What does kmain() do?}\newline
\textbf{Answer: Kernel main. (Boot up, then fork the menu thread, wait for a reboot request, and then shut down.)
	from kern/main/main.c
	}\\ \\
\textit{Q22. Is it OK to initialise the thread system before the scheduler? Why (not)?}\newline
\textbf{Answer: Yes. The scheduler bootstrap just creates the run queue, and the thread bootstrap just initializes the first thread.}\\ \\
\textit{Q23. What is a zombie?}\newline
\textbf{Answer: “Zombies are threads/processes that have exited but not been fully deleted yet.”
	from kern/thread/thread.c
	}\\ \\
\textit{Q24. How large is the initial run queue?}\newline
\textbf{Answer: The initial run queue is 32.}\\ \\
\textit{Q25. What does a device name in OS/161 look like?}\newline
\textbf{Answer: The name of a device is always device:, such as lhd0:
	from kern/vfs/device.c
	}\\ \\
\textit{Q26. What does a raw device name in OS/161 look like?}\newline
\textbf{Answer: The name with raw appended, such as lhd0raw,
	from kern/vfs/vfslist.c 
	}\\ \\
\textit{Q27. What lock protects the vnode reference count?}\newline
\textbf{Answer: vn\_countlock protects the vnode reference count from kern/vfs/vnode.c
		}\\ \\
\textit{Q28. What device types are currently supported?}\newline
\textbf{Answer: The device types currently supported are block and character devices.
	from kern/vfs/device.c
	}\\ \\
......................
%\section{conclusion}
\end{document}
